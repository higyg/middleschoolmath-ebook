% !Mode:: "TeX:UTF-8"


\begin{defproblem}{4NJ-01-01}%
\begin{onlyproblem}%
$a\divideontimes b$表示$a$与$b$的乘积减去$a$与$b$的和的差,即$a\divideontimes b=a\times b-(a+b)$。
 
例如: 当$a=3$,$b=4$,$3\divideontimes 4=3\times 4-(3+4)$。 
  
求$5\divideontimes 7=$?

\end{onlyproblem}%
\begin{onlysolution}%
$5\divideontimes 7=5\times7-(5+7)=23$
\end{onlysolution}%
\end{defproblem}


\begin{defproblem}{4NJ-01-02}%
\begin{onlyproblem}%
定义新运算为$a\triangle b=(a+1)\div b$,$a\oplus b=a+2b$。

例如:当$a=5$,$b=3$时,$5\triangle 3=(5+1)\div 3=2$。 

求:$6\oplus (3\triangle 4)=$? 

\end{onlyproblem}%
\begin{onlysolution}%
$\begin{array}[t]{r@{~}l}
&\quad6\oplus (3\triangle 4)\\
&=6\oplus [(3+1)\div4]\\
&=6\oplus1\\
&=6+2\times1\\
&=8
\end{array}
$

\end{onlysolution}%
\end{defproblem}


\begin{defproblem}{4NJ-01-03}%
\begin{onlyproblem}%
定义新运算$a\odot b=3a-2b$

(1)计算:$(8\odot 7)\odot 9$。 

(2)已知:$x\odot (4\odot 1)=7$,求:$x=$?


\end{onlyproblem}%
\begin{onlysolution}%
(1)$\begin{array}[t]{r@{~}l}
&\quad(8\odot 7)\odot 9\\
&=(3\times8-2\times7)\odot9\\
&=10\odot9\\
&=3\times10-2\times9\\
&=12
\end{array}
$
(2)$\begin{array}[t]{r@{~}l}
&\quad x\odot (4\odot 1)\\
&=x\odot [3\times4-2\times1]\\
&=x\odot 10\\
&=3x-2\times10\\
&=3x-20
\end{array}
$
所以
$\begin{array}[t]{r@{~}l}
3x-20&=7\\
3x&=27\\
x&=9
\end{array}
$

\end{onlysolution}%
\end{defproblem}




\begin{defproblem}{4NJ-01-04}%
\begin{onlyproblem}%
设$a$、$b$表示两个不同的数,规定$a\Delta b=4\times a-3\times b$。求$5\Delta 6$?

\end{onlyproblem}%
\begin{onlysolution}%
$\begin{array}[t]{r@{~}l}
&\quad5\Delta 6\\
&=4\times5-3\times6\\
&=2
\end{array}
$
\end{onlysolution}%
\end{defproblem}






\begin{defproblem}{4NJ-01-05}%
\begin{onlyproblem}%
定义新的运算为$a\ominus b=a\times b+a+b$。求$8\ominus 9$、$(1\ominus 2)\ominus 3$?

\end{onlyproblem}%
\begin{onlysolution}%
(1)$\begin{array}[t]{r@{~}l}
&\quad8\ominus 9\\
&=8\times9+8+9\\
&=72+17\\
&=89
\end{array}
$
\qquad
$\begin{array}[t]{r@{~}l}
&\quad (1\ominus 2)\ominus 3\\
&=(1\times 2+1+2)\ominus 3\\
&=5\ominus 10\\
&=5\times10+5+10\\
&=50+15\\
&=65\\
\end{array}
$

\end{onlysolution}%
\end{defproblem}






\begin{defproblem}{4NJ-01-06}%
\begin{onlyproblem}%
设$a$、$b$表示两个不同的数,规定$a\oplus b=3\times a-2\times b$,如果已知$4\oplus b=2$。求$b=$?
\end{onlyproblem}%
\begin{onlysolution}%
(1)$\begin{array}[t]{r@{~}l}
&\quad 4\oplus b\\
&=3\times4-2\times b\\
&=12-2b
\end{array}
$
\quad
所以$\begin{array}[t]{r@{~}l}
12-2b&=2\\
12&=2+2b\\
10&=2b\\
2b&=10\\
b&=5\\
\end{array}
$
\end{onlysolution}%
\end{defproblem}






\begin{defproblem}{4NJ-01-07}%
\begin{onlyproblem}%
定义新运算为$a\nabla b=(a+1)\div b$。求$2\nabla (3\nabla 4)$的值?
\end{onlyproblem}%
\begin{onlysolution}%
$\begin{array}[t]{r@{~}l}
&\quad 2\nabla (3\nabla 4)\\
&=2\nabla [(3+1)\div 4]\\
&=2\nabla 1\\
&=(2+1)\div 1\\
&=3
\end{array}
$
\end{onlysolution}%
\end{defproblem}




\begin{defproblem}{4NJ-01-08}%
\begin{onlyproblem}%
规定运算``$\bigstar$''为: 

若$a>b$,则$a\bigstar b=a+b$;

若$a=b$,则$a\bigstar b=a-b+1$;           

若$a<b$,则$a\bigstar b=a\times b$。

那么$(2\bigstar 3)+(4\bigstar 4)+(7\bigstar 5)$的值? 

\end{onlyproblem}%
\begin{onlysolution}%
定义新运算综合巩固

$\begin{array}[t]{r@{~}l}
&\quad (2\bigstar 3)+(4\bigstar 4)+(7\bigstar 5)\\
&=(2\times3)+(4-4+1)+(7+5)\\
&=6+1+12\\
&=19\\
\end{array}
$
\end{onlysolution}%
\end{defproblem}



\begin{defproblem}{4NJ-01-09}%
\begin{onlyproblem}%
我们规定:$A\bigcirc B$表示$A$、$B$中较大的数,$A\triangle B$表示$A$、$B$中较小的数。

求:$(10\triangle 8-6\triangle 5)\times(11\bigcirc13+15\triangle 20)$的值?


\end{onlyproblem}%
\begin{onlysolution}%
定义新运算综合巩固

$\begin{array}[t]{r@{~}l}
&\quad (10\triangle 8-6\triangle 5)\times(11\bigcirc13+15\triangle 20)\\
&=(8-5)\times(13+15)\\
&=3\times28\\
&=84\\
\end{array}
$
\end{onlysolution}%
\end{defproblem}



\begin{defproblem}{4NJ-01-10}%
\begin{onlyproblem}%
已知:$10\triangle 3=14$, $8\triangle 7=2$,$6\triangle 3=6$,$8\triangle 2=12$,根据这几个算式找规律
如果$13\triangle x=8$,那么$x=$?



\end{onlyproblem}%
\begin{onlysolution}%
定义新运算综合巩固

$10\triangle 3=(10-3)\times 2=14$、$8\triangle 7=(8-7)\times 2=2$、$6\triangle 3=(6-3)\times 2=6$

规律是:$a\triangle b=(a-b)\times2$


$\begin{array}[t]{r@{~}l}
13\triangle x
&=(13-x)\times2\\
(13-x)\times2&=8\\
13-x&=4\\
x&=9\\
\end{array}
$
\end{onlysolution}%
\end{defproblem}





\begin{defproblem}{4NJ-01-11}%
\begin{onlyproblem}%
``$\odot$''表示一种新的运算符号,已知:$2\odot3=2+3+4$,$7\odot2=7+8$,
$3\odot 5=3+4+5+6+7$,……按此规则:
如果$n \odot 8=68$,那么,$n=$?


\end{onlyproblem}%
\begin{onlysolution}%
定义新运算综合巩固

$\odot$表示几个连续自然数之和,$\odot$前面的数表示第一个加数,$\odot$后面的数表示加数的个数,于是$n + (n + 1) + (n + 2) +  \cdots  + (n + 7) = 68$,

即
$\begin{array}[t]{r@{~}l}
8n+28&=68\\
8n&=40\\
n&=5\\
\end{array}
$
\end{onlysolution}%
\end{defproblem}





\begin{defproblem}{4NJ-01-12}%
\begin{onlyproblem}%
任意的数$a$,$b$,定义:$f(a)=2a+1$,$g(b)=b\times b$。

(1) 求$f(5)-g(3)$的值;

(2) 求$f(g(2))+g(f(2))$的值;

(3) 已知$f(x+1)=21$,求$x$的值。


\end{onlyproblem}%
\begin{onlysolution}%
定义新运算综合巩固

(1)$f(5)-g(3)=(2\times 5+1)-(3\times 3)=2$;

(2)$f(g(2))+g(f(2))=f(2\times 2)+g(2\times 2+1)=f(4)+g(5)=(2\times 4+1)+(5\times 5)=34$;

(3)$f(x+1)=2(x+1)+1=2x+3$,由$f(x+1)=21$,知$2x+3=21$,解得$x=9$。

\end{onlysolution}%
\end{defproblem}








