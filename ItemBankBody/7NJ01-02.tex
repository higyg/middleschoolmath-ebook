% !Mode:: "TeX:UTF-8"
% 七年级上学期第一单元几何体展开图

\begin{defproblem}{7NJ-02-01}%
\begin{onlyproblem}%
下列四个图形中,是三棱柱的表面展开图的是
\begin{center}
\includegraphics[width=2cm]{7NJ01-01-20190803-1.jpg}
\includegraphics[width=2cm]{7NJ01-01-20190803-2.jpg}
\includegraphics[width=2cm]{7NJ01-01-20190803-3.jpg}
\includegraphics[width=2cm]{7NJ01-01-20190803-4.jpg}
\end{center}
\end{onlyproblem}%
\begin{onlysolution}%
\begin{solution}%%
B
\end{solution}%
\end{onlysolution}%
\end{defproblem}





\begin{defproblem}{7NJ-02-02}%
\begin{onlyproblem}%
下面6个图形是正方体的表面展开图的有
\begin{center}
\includegraphics[width=4cm]{7NJ01-01-20190803-5.jpg}
\includegraphics[width=4cm]{7NJ01-01-20190803-6.jpg}
\includegraphics[width=4cm]{7NJ01-01-20190803-7.jpg}
\end{center}
\xx
{2个}
{3个}
{4个}
{5个}
\end{onlyproblem}%
\begin{onlysolution}%
\begin{solution}%%
B
\end{solution}%
\end{onlysolution}%
\end{defproblem}




\begin{defproblem}{7NJ-02-03}%
\begin{onlyproblem}%
从如图的纸板上11个无阴影的正方形中选1个(将其余10个都剪去),与图中5个有阴影的正方形折成一个正方体,不同的选法有(    ) 
\begin{center}
\includegraphics[width=3cm]{7NJ01-01-20190803-8.jpg}
\end{center}

\xx
{6种}
{5种}
{4种}
{3种}

\end{onlyproblem}%
\begin{onlysolution}%
\begin{solution}%%
C
\end{solution}%
\end{onlysolution}%
\end{defproblem}




\begin{defproblem}{7NJ-02-04}%
\begin{onlyproblem}%
下列四个选项的图形折叠后,能得到如图所示的正方体的是(    ) 
\begin{center}
\includegraphics[width=1cm]{7NJ01-01-20190803-9.jpg}
\includegraphics[width=10cm]{7NJ01-01-20190803-10.jpg}
\end{center}


\end{onlyproblem}%
\begin{onlysolution}%
\begin{solution}%%
C
\end{solution}%
\end{onlysolution}%
\end{defproblem}



\begin{defproblem}{7NJ-02-05}%
\begin{onlyproblem}%
将“创建文明城市”六个字分别写在一个正方体的六个面上,这个正方体的表面展开图如图所示,那么在这个正方体中,和“创”相对的字是(    ) 
\begin{center}
\includegraphics[width=3cm]{7NJ01-01-20190803-11.jpg}
\end{center}

\xx
{文}
{明}
{城}
{市}

\end{onlyproblem}%
\begin{onlysolution}%
\begin{solution}%%
B
\end{solution}%
\end{onlysolution}%
\end{defproblem}




\begin{defproblem}{7NJ-02-06}%
\begin{onlyproblem}%
如图,是一个正方体的表面展开图,在正方体中写有“心”字的那一面的相对面的字是(    ) 
\begin{center}
\includegraphics[width=2cm]{7NJ01-01-20190803-12.jpg}
\end{center}

\xx
{祝}
{你}
{事}
{成}

\end{onlyproblem}%
\begin{onlysolution}%
\begin{solution}%%
B
\end{solution}%
\end{onlysolution}%
\end{defproblem}



\begin{defproblem}{7NJ-02-07}%
\begin{onlyproblem}%
小明为了鼓励芦山地震灾区的学生早日走出阴影,好好学习,制作了一个正方体礼盒(如图).礼盒每个面上各有一个字,连起来组成“芦山学子加油”,其中“芦”的对面是“学”,“加”的对面是“油”,则它的表面展开图可能是(    ) 
\begin{center}
\includegraphics[width=1cm]{7NJ01-01-20190803-13.jpg}
\includegraphics[width=10cm]{7NJ01-01-20190803-14.jpg}
\end{center}


\end{onlyproblem}%
\begin{onlysolution}%
\begin{solution}%%
C
\end{solution}%
\end{onlysolution}%
\end{defproblem}




\begin{defproblem}{7NJ-02-08}%
\begin{onlyproblem}%
六个面分别标有“我”、“是”、“初”、“一”、“学”、“生”的正方体有三种不同放置方式,则“是”和“学”的相对面分别是(    ) 
\begin{center}
\includegraphics[width=6cm]{7NJ01-01-20190803-15.jpg}
\end{center}

\xx
{“生”和“一”}
{“初”和“生”}
{“初”和“一”}
{“生”和“初”}   


\end{onlyproblem}%
\begin{onlysolution}%
\begin{solution}%%
C
\end{solution}%
\end{onlysolution}%
\end{defproblem}



\begin{defproblem}{7NJ-02-09}%
\begin{onlyproblem}%
一个小立方块的六面分别标有字母A,B,C,D,E,F,如图是从三个不同方向看到的情形,则A,B,E的相对面分别是(    ) 
\begin{center}
\includegraphics[width=5cm]{7NJ01-01-20190803-16.jpg}
\end{center}

\xx
{E,D,F}
{E,F,D}
{F,D,E}
{F,D,C}


\end{onlyproblem}%
\begin{onlysolution}%
\begin{solution}%%
D
\end{solution}%
\end{onlysolution}%
\end{defproblem}






