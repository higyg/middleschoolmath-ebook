% !Mode:: "TeX:UTF-8"
% \textbf{计算专题------实数混合运算}

\begin{defproblem}{T18-A01-01}%
\begin{onlyproblem}%
(9分)如图,在$\triangle ABC$中,$\angle C=90^{\circ}$,点$O$在$AC$上,以$OA$为半径的$\odot O$交$AB$于点$D$,$BD$的垂直平分线交$BC$于点$E$,交$BD$于点$F$,连接$DE$.


(1)判断直线$DE$与$\odot O$的位置关系,并说明理由;

(2)若$AC=6$,$BC=8$,$OA=2$,求线段$DE$的长.

\vspace*{2\baselineskip}
\tabto{0.61\columnwidth}
\includegraphics[scale=1]{\Newcurrentlabel.pdf}


\end{onlyproblem}%
\begin{onlysolution}%
\includegraphics[scale=.5]{\Newcurrentlabel.jpg}

\end{onlysolution}%
\end{defproblem}



\begin{defproblem}{T18-A01-02}%
\begin{onlyproblem}%
如图,已知$AB$是$\odot O$的直径,$PC$切$\odot O$于点$P$,过$A$作直线$AC\bot PC$,交$\odot O$于另一点$D$,连接$PA$,$PB$.

(1)求证:$AP$平分$\angle CAB$;

(2)若$P$是直径$AB$上方半圆弧上一动点,$\odot O$的半径为2,则:

①当弦$AP$的长是{\_}{\_}{\_}{\_}{\_}{\_}{\_}时,以$A$,$O$,$P$,$C$为顶点的四边形是正方形;

②当$\widehat{AP}$的长度是{\_}{\_}{\_}{\_}{\_}{\_}{\_}时,以$A$,$D$,$O$,$P$为顶点的四边形是菱形.

\vspace*{-\baselineskip}
\tabto{0.75\columnwidth}
\includegraphics[scale=.82]{\Newcurrentlabel.pdf}


\end{onlyproblem}%
\begin{onlysolution}%
\begin{center}
\includegraphics[scale=.82]{\Newcurrentlabel.pdf}
\end{center}
\end{onlysolution}%
\end{defproblem}


