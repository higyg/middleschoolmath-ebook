\begin{center}
{\kaishu\zihao{-2}计算专题训练}
\end{center}
% \renewcommand\baselinestretch{1.3}
% \setlist{noitemsep}


{\heiti 学习目标}
\begin{enumerate}
\item
熟练进行整式和分式的相关运算;

\item
根据题目要求, 能准确地进行代数式求值.
\end{enumerate}

{\heiti 知识点睛}

\begin{enumerate}
% \item 基本运算操作规程

% 看结构,分部分;依法则,不跳步;警异常,巧检验.

% 检验原则:{\_}{\_}{\_}{\_}{\_}{\_}{\_}{\_}{\_}{\_}{\_}{\_}{\_}{\_}{\_}{\_}{\_}{\_}{\_}{\_}.


\item 解方程的依据是{\_}{\_}{\_}{\_}{\_}{\_}{\_}{\_}{\_}{\_}{\_}{\_}{\_};解多元方程组的基本思路是{\_}{\_}{\_}{\_}{\_}{\_}{\_}{\_}{\_};\\
解高次方程的基本思路是{\_}{\_}{\_}{\_}{\_}{\_}{\_}.\\
分式方程先转化为{\_}{\_}{\_}{\_}{\_}{\_}{\_}{\_}{\_}{\_}{\_}{\_}{\_}{\_}{\_},结果必须{\_}{\_}{\_}{\_}{\_}{\_}{\_}{\_}{\_}{\_}{\_}.
\item 解不等式的依据是{\_}{\_}{\_}{\_}{\_}{\_}{\_}{\_}{\_}{\_}{\_}{\_}{\_}{\_}{\_}{\_}{\_}{\_}{\_}{\_}{\_};\\
解不等式的基本思路是转化为{\_}{\_}{\_}{\_}{\_}{\_}{\_}{\_}{\_}{\_}{\_}{\_}{\_}{\_}{\_}{\_};\\
求解集时,常借助{\_}{\_}{\_}{\_}{\_}{\_}{\_}找公共部分.
\item 综合运算问题的处理思路:
①分析问题,明确目标;
②观察结构,分析特征;
③边运算,边调整.

注:显性条件:正整数、负整数、无解、增根等.

隐性条件:由定义、性质、指代不明等造成的范围限制.

如:$\dfrac{1}{a}$中$a\ne $0;$\sqrt a $中$a\ge $0;不等式$ax<b$的解集为$x>1$,隐含$a<0$,且$a$与$b$同号;一元二次方程使用$\Delta $时,要保证二次项系数不为0.
\end{enumerate}

% \noindent%
{\heiti 精讲精练}

\begin{enumerate}
\foreachproblem[T16-1]{\item\label{prob:\thisproblemlabel}\thisproblem}\vfill\null
\foreachproblem[T16-2]{\item\label{prob:\thisproblemlabel}\thisproblem}\vfill\null\newpage
\foreachproblem[T16-3]{\item\label{prob:\thisproblemlabel}\thisproblem}\vfill\null
\foreachproblem[T16-4]{\item\label{prob:\thisproblemlabel}\thisproblem}\vfill\null\newpage
\foreachproblem[T16-5]{\item\label{prob:\thisproblemlabel}\thisproblem}\vfill\null
\foreachproblem[T16-6]{\item\label{prob:\thisproblemlabel}\thisproblem}\vfill\null\newpage
\foreachproblem[T16-7]{\item\label{prob:\thisproblemlabel}\thisproblem}\vfill
\foreachproblem[T16-8]{\item\label{prob:\thisproblemlabel}\thisproblem}\vfill\null

\end{enumerate}




\newpage
\showanswers
\section*{参考答案}
\begin{enumerate}
\foreachdataset{\thisdataset}{%
\foreachproblem[\thisdataset]{\item[\ref{prob:\thisproblemlabel}]\thisproblem}}
\end{enumerate}