\documentclass[10pt,twoside,fleqn]{ctexart}
\usepackage{geometry}
\usepackage{examanswersheet,caption2,pifont}
\Nomifengxian
\usepackage{enumitem,bbding}
% \setenumerate[1]{itemsep=5pt,partopsep=0pt,parsep=\parskip,topsep=0pt,leftmargin=2em}
\setitemize[1]{itemsep=0pt,partopsep=0pt,parsep=\parskip,topsep=0pt}
\setdescription{itemsep=0pt,partopsep=0pt,parsep=\parskip,topsep=0pt}
\setlist[enumerate,1]{label=\arabic*.,leftmargin=2em,itemsep=0pt}
\setlist[enumerate,2]{label=(\arabic*),leftmargin=1em,itemsep=0pt}
\newlist{choices}{enumerate}{3}
\setlist[choices]{noitemsep,label=\Alph*.}
\newlist{questions}{enumerate}{3}
\setlist[questions]{noitemsep}
\setlist[questions,1]{label=\arabic*.}
\setlist[questions,2]{label=(\arabic*)}

\usepackage{mathptmx}
\punctstyle{banjiao}
\usepackage{subfigure}
\graphicspath{{figure/}{ItemBankfigures/}}
\usepackage{epstopdf}
\epstopdfsetup{suffix={}}


\makeatletter
\providecommand*\input@path{}
\newcommand\addinputpath[1]{
\expandafter\def\expandafter\input@path
\expandafter{\input@path{#1}}}
\addinputpath{ItemBankBody/}
\makeatother

\renewcommand*{\le}{\leqslant}
\renewcommand*{\ge}{\geqslant}

\usepackage[colorlinks,linkcolor=blue,anchorcolor=black,citecolor=blue]{hyperref}

\xeCJKsetcharclass{`①}{`⑩}{1}
% \setlength\lineskip{6pt}
\setlength\lineskiplimit{1pt}
\setlength\lineskip{3pt}

\usepackage[draft]{probsoln}
% \usepackage[final]{probsoln}
\SetStartYear{2019}
\PSNrandseed{\GetStartYear}
\PSNrandseed{0502}
% \showanswers
\ClearUsedFile{excluded}
% \loadallproblems{ItemBankBody/T16-01}
% \loadallproblems{ItemBankBody/T16-02}
% \loadallproblems{ItemBankBody/T16-03}
% \loadallproblems{ItemBankBody/T16-04}

% \loadselectedproblems[T16-A01]{T16-A01-01}{T16-01}
% \loadselectedproblems[T16-A02]{fenshi:huajianqiuzhi4}{fenshi}
% \loadselectedproblems[T16-B01]{T16-A01-07}{T16-01}
% \loadselectedproblems[T16-B02]{T16-A02-02}{T16-02}
% \loadselectedproblems[T16-C01]{T16-A03-02}{T16-03}
% \loadselectedproblems[T16-C02]{T16-A03-05}{T16-03}
% \loadselectedproblems[T16-D01]{T16-A05-04}{T16-05}
% \loadselectedproblems[T16-E01]{T16-A06-02}{T16-06}

% \loadselectedproblems[T16-1]{T16-A01-01}{T16-01}
% \loadselectedproblems[T16-2]{T16-A01-02}{T16-01}
% \loadselectedproblems[T16-3]{T16-A02-01}{T16-02}
% \loadselectedproblems[T16-4]{T16-A03-01}{T16-03}
% \loadselectedproblems[T16-5]{T16-A03-02}{T16-03}
% \loadselectedproblems[T16-6]{T16-A05-01}{T16-05}
% \loadselectedproblems[T16-7]{T16-A04-01}{T16-04}
% \loadselectedproblems[T16-8]{T16-A06-01}{T16-06}

\loadselectedproblems[20190603T16-1]{T16-A02-06}{T16-02}
\loadselectedproblems[20190603T16-2]{T16-A03-03}{T16-03}
\loadselectedproblems[20190603T16-3]{T16-A04-02}{T16-04}
\loadselectedproblems[20190603T16-4]{T16-A05-02}{T16-05}
\loadselectedproblems[20190603T16-5]{T16-A05-04}{T16-05}
\loadselectedproblems[20190603T16-6]{T16-A06-01}{T16-06}
\loadselectedproblems[20190603T16-7]{T16-A06-04}{T16-06}
% \loadselectedproblems[20190603T16-8]{T16-A06-01}{T16-06}


% \loadselectedproblems[fenshi:huajianqiuzhi9]{fenshi:huajianqiuzhi9}{fenshi}

\begin{document}
% \begin{multicols}{3}
\begin{center}
{\kaishu\zihao{-2}计算专题训练}
\end{center}
% \renewcommand\baselinestretch{1.3}
% \setlist{noitemsep}


{\heiti 学习目标}
\begin{enumerate}
\item
熟练进行整式和分式的相关运算;

\item
根据题目要求, 能准确地进行代数式求值.
\end{enumerate}

{\heiti 知识点睛}

\begin{enumerate}
% \item 基本运算操作规程

% 看结构,分部分;依法则,不跳步;警异常,巧检验.

% 检验原则:{\_}{\_}{\_}{\_}{\_}{\_}{\_}{\_}{\_}{\_}{\_}{\_}{\_}{\_}{\_}{\_}{\_}{\_}{\_}{\_}.


\item 解方程的依据是{\_}{\_}{\_}{\_}{\_}{\_}{\_}{\_}{\_}{\_}{\_}{\_}{\_};解多元方程组的基本思路是{\_}{\_}{\_}{\_}{\_}{\_}{\_}{\_}{\_};\\
% 解高次方程的基本思路是{\_}{\_}{\_}{\_}{\_}{\_}{\_}.\\
分式方程先转化为{\_}{\_}{\_}{\_}{\_}{\_}{\_}{\_}{\_}{\_}{\_}{\_}{\_}{\_}{\_},结果必须{\_}{\_}{\_}{\_}{\_}{\_}{\_}{\_}{\_}{\_}{\_}.
\item 解不等式的依据是{\_}{\_}{\_}{\_}{\_}{\_}{\_}{\_}{\_}{\_}{\_}{\_}{\_}{\_}{\_}{\_}{\_}{\_}{\_}{\_}{\_};\\
解不等式的基本思路是转化为{\_}{\_}{\_}{\_}{\_}{\_}{\_}{\_}{\_}{\_}{\_}{\_}{\_}{\_}{\_}{\_};\\
求解集时,常借助{\_}{\_}{\_}{\_}{\_}{\_}{\_}找公共部分.
\item 综合运算问题的处理思路:
①分析问题,明确目标;
②观察结构,分析特征;
③边运算,边调整.

% 注:显性条件:正整数、负整数、无解、增根等.

% 隐性条件:由定义、性质、指代不明等造成的范围限制.

% 如:$\dfrac{1}{a}$中$a\ne $0;$\sqrt a $中$a\ge $0;不等式$ax<b$的解集为$x>1$,隐含$a<0$,且$a$与$b$同号;一元二次方程使用$\Delta $时,要保证二次项系数不为0.
\end{enumerate}

% \noindent%
{\heiti 精讲精练}

\begin{enumerate}
\foreachproblem[20190603T16-1]{\item\label{prob:\thisproblemlabel}\thisproblem}\vfill\null
\foreachproblem[20190603T16-2]{\item\label{prob:\thisproblemlabel}\thisproblem}\null\newpage
\foreachproblem[20190603T16-3]{\item\label{prob:\thisproblemlabel}\thisproblem}\vfill\null
\foreachproblem[20190603T16-4]{\item\label{prob:\thisproblemlabel}\thisproblem}\vfill\null\newpage
\foreachproblem[20190603T16-5]{\item\label{prob:\thisproblemlabel}\thisproblem}\vfill\null
\foreachproblem[20190603T16-6]{\item\label{prob:\thisproblemlabel}\thisproblem}\vfill\null\newpage
\foreachproblem[20190603T16-7]{\item\label{prob:\thisproblemlabel}\thisproblem}\vfill
% \foreachproblem[T16-8]{\item\label{prob:\thisproblemlabel}\thisproblem}\vfill\null

\end{enumerate}


% \noindent%
% {\heiti 精讲}

% \begin{enumerate}
% \foreachproblem[T16-C01]{\item{(\kaishu\zihao{-5}{错误原因分析})}\label{prob:\thisproblemlabel}\thisproblem}
% \vfill\newpage

% \foreachproblem[T16-A01]{\item{(\kaishu\zihao{-5}{取值说理,与不等式结合})}\label{prob:\thisproblemlabel}\thisproblem}
% \vfill\null

% \foreachproblem[T16-B01]{\item{(\kaishu\zihao{-5}{取值说理,与方程结合})}\label{prob:\thisproblemlabel}\thisproblem}
% \vfill\null




% \foreachproblem[T16-E01]{\item{(\kaishu\zihao{-5}{与其他知识的综合})}\label{prob:\thisproblemlabel}\thisproblem}
% \vfill\null

% \end{enumerate}


% \newpage
% \noindent%
% {\heiti 精练}

% \begin{enumerate}
% \foreachproblem[T16-C02]{\item\label{prob:\thisproblemlabel}\thisproblem}
% % \vfill

% \foreachproblem[T16-B02]{\item\label{prob:\thisproblemlabel}\thisproblem}
% \vfill\null
% \newpage
% \foreachproblem[T16-A02]{\item\label{prob:\thisproblemlabel}\thisproblem}
% \vfill\null

% % {(\kaishu\zihao{-5}{无取值说理、直接代入})}
% \foreachproblem[T16-D01]{\item\label{prob:\thisproblemlabel}\thisproblem}
% \vfill\null

% % \foreachproblem[T16-D01]{\item{(\kaishu\zihao{-5}{无取值说理、直接代入})}\label{prob:\thisproblemlabel}\thisproblem}
% % \vfill\null
% % \foreachproblem[T16-E01]{\item{(\kaishu\zihao{-5}{与其他知识的综合})}\label{prob:\thisproblemlabel}\thisproblem}
% % \vfill\null

% \end{enumerate}





\newpage
\showanswers
\section*{参考答案}
\begin{enumerate}
\foreachdataset{\thisdataset}{%
\foreachproblem[\thisdataset]{\item[\ref{prob:\thisproblemlabel}]\thisproblem}}
\end{enumerate}




% \raggedcolumns
% \flushcolumns

% \end{multicols}
\end{document}